\section*{Závěr}
\label{sec:conc}
\addcontentsline{toc}{section}{\nameref{sec:conc}}

Vysvětlil jsem hlavní problémy kvantové mechaniky, které upozorňují na její vnitřní nesrovnalosti a důvody, proč její doplnění, nebo nahrazení fundamentálnější teorií vyžaduje porušení Statistické Nezávislosti, předpokladu, kterému se často chybně říká Svobodná Vůle. Vyložil jsem původ teorií s touto vlastností, běžně označovaných jako superdeterministické, v Bellově nerovnosti. Extenzivně jsem popsal hlavní rysy těchto teorií. Vyvrátil jsem hlavní námitky vznesené proti Superdeterminismu, které jasně vychází z naší intuice, která je založena na makroskopických procesech, které pozorujeme našimi smysly a na závěr jsem navrhl jak bychom měli dále postupovat ve vývoji Superdeterminismu.

Od objevu kvantové mechaniky se věda ubírá cestou unifikace dvou neslučitelných teorií, obecné relativity a kvantové mechaniky (přesněji standardního modelu částicové fyziky), pomocí kvantizace gravitace. Tento přístup, soustředěný na změnu gravitace k tomu, aby mohla být kvantizována, je následován už přibližně 70 let, ale přesto nevykázal mnoho slibných výsledků. Z tohoto důvodu si myslím, že další krok vpřed by měl být zpět k přehodnocení kvantizace, k Bellově nerovnosti a možnosti, že chování kvantových částic závisí na tom, jaké měření se uskuteční; možnosti která byla zamítnuta kvůli neoprávněným obavám a špatnému názvosloví.