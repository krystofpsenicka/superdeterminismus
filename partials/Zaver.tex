\section*{Závěr}
\label{sec:conc}
\addcontentsline{toc}{section}{\nameref{sec:conc}}

V práci jsem předložil a vysvětlil hlavní problémy kvantové mechaniky, které upozorňují na její vnitřní nesrovnalosti, a důvody, proč její doplnění, nebo nahrazení fundamentálnější teorií vyžaduje porušení Statistické Nezávislosti, předpokladu, kterému se často chybně říká Svobodná Vůle. Vyložil jsem původ teorií s touto vlastností, běžně označovaných jako superdeterministické, v Bellově nerovnosti. Extenzivně jsem popsal hlavní rysy těchto teorií a vyvrátil jsem hlavní námitky vznesené proti Superdeterminismu, které jasně vychází z naší intuice, která je založena na makroskopických procesech pozorovaných našimi smysly. Závěrem jsem navrhl, jak bychom měli ve vývoji Superdeterminismu dále postupovat.

Od objevu kvantové mechaniky se věda ubírá cestou unifikace dvou neslučitelných teorií, obecné relativity a kvantové mechaniky (přesněji standardního modelu částicové fyziky), pomocí kvantizace gravitace. Tento přístup, soustředěný na změnu gravitace k tomu, aby mohla být kvantizována, je následován už přibližně 70 let, přesto nevykázal mnoho slibných výsledků. Proto si myslím, že další postup by měl být návratem k přehodnocení kvantizace, k Bellově nerovnosti a k možnosti, že chování kvantových částic závisí na uskutečněném měření; k možnosti, která byla zamítnuta kvůli neoprávněným obavám a špatnému názvosloví.