\section{Protiargumenty}
(Zpracováno podle článku \cite{supdet:rethink})
\subsection{Bellovy testy}


Nobelovu cenu za fyziku za rok 2022 dostala trojice fyziků Alain Aspect, Anton Zeilinger a John Clauser. Většina médií milně uvádí, že tato trojice dostala Nobelovu cenu za dokázání kvantového provázání částic pomocí mnoha extenzivních testů Bellovy nerovnosti. Žádný z těchto experimentů nedokazuje kvantové provázání, ani nemožnost Superdeterminismu.

\subsubsection{Kosmické Bellovy testy}
V Kosmických Bellových testech (\cite{CosBTest:1}; \cite{CosBTest:2}) jsou nastavení měření určena podle přesné vlnové délky světla přicházejícího z kvazarů (velmi vzdálených objektů, které byly kauzálně odděleny v čas emise fotonů). Tento experiment je velmi pozoruhodný a je hoden Nobelovy ceny, ale nemůže vyloučit Superdeterminismus. Pouze říká, že korelace pozorovené v Bellových testech nemohly být lokálně způsobeny událostmi ve vzdálené minulosti (v případě Kosmických testů jsou těmito událostmi emise fotonů až miliardy let v minulosti, chvíli po vzniku vesmíru). Porušení Bellovy nerovnosti nám pouze říká, že jeden z předpokladů Bellovy nerovnosti byl porušen. Žádný Bellův test nemůže určit, jaký předpoklad byl porušen.

Domněnka, že takové testy říkají něco o nemožnosti Superdeterminismu, vycházejí z předpokladu, že stav blízký realizovanému stavu (např. kdyby světlo vyzařované vzdálenými kvazary mělo trochu jinou vlnovou délku) je povolen přírodními zákony a je pravděpodobný. V superdeterministické teorii by ale tato malá změna vytvořila extrémně nepravděpodobný stav. Změna vlnové délky světla z kvazarů by mohla vyžadovat změnu jinde na hyperporvchu daného momentu, která by vedla k rozhodnutí experimentátora nepoužít světlo z daných kvazarů ve svém experimentu.
\clearpage

\subsubsection{Velký Bellův test}
Autoři Velkého Bellova testu si uvědomili problém ve vytváření náhodnosti v nastavení měření k vyloučení korelace mezi skrytými proměnnými a nastavením měření (porušení Statistické Nezávislosti):
\begin{quote}
    \emph{A Bell test requires spatially distributed entanglement, fast and high-efficiency detection and unpredictable measurement settings. Although technology can satisfy the first two of these requirements the use of physical devices to choose settings in a Bell test involves making assumptions about the physics that one aims to test. Bell himself noted this weakness in using physical setting choices and argued that human ‘free will’ could be used rigorously to ensure unpredictability in Bell tests. Here we report a set of local-realism tests using human choices, which avoids assumptions about predictability in physics.} - \cite{BigBTest}
\end{quote}

Velký Bellův test proběhl tak, že 100 000 lidských účastníků hrálo online videohru, čímž bylo vygenerováno 97 347 490 binárních voleb, které byly poslány do 12 laboratoří na pěti kontinentech, kde bylo provedeno 13 experimentů testujících Bellovu nerovnost pomocí fotonů, jednotlivých atomů, atomových souborů a supravodivých zařízení. Tímto experimentem byla údajně \uv{uzavřena mezera svobodné volby (možnost, že nastavení měření jsou ovlivněna skrytými proměnnými, a tím korelována s vlastnostmi částic)}. Je jednoduché poznat problém v závěru, ke kterému dospěli autoři tohoto článku. Experimentátoři předpokládají, že vstup od lidí je na rozdíl od vstupu vytvářeného fyzickými přístroji náhodný, a tím předpokládají fundamentální rozdíl mezi lidmi a fyzickými systémy (schopnost svobodné volby). Experiment předpokládá svobodu vůle účastníků při hraní videohry k vytvoření náhodnosti v nastavení měření a k následnému dokázání svobodné vůle. Svobodnou vůli nelze dokázat tím, že předpokládáme svobodnou vůli.

Ve skutečnosti moc nezáleží na detailech těchto experimentů. Musíme vzít na vědomí, že změření porušení Bellovy nerovnosti, ať je experiment jakkoliv komplexní, nemůže určit, které předpoklady nerovnosti byly porušeny.

\clearpage

\subsection{Svobodná vůle}
Předpoklad Statistické Nezávislosti se většinou označuje jako předpoklad svobodné vůle, protože může být interpretován jako svoboda experimentátora vybrat si nastavení měření nezávisle na skrytých proměnných.

Je důležité rozlišovat dvě převládající definice svobodné vůle:
\begin{enumerate}
    \item \textbf{Libertariánská (Li):} možnost jednat jinak.
    \item \textbf{Kompatibilistická (Ko):} nepřítomnost omezení, která by člověku bránila dělat to, co si přeje.
\end{enumerate}

Statistická Nezávislost se opírá o Li, jelikož obojí vychází z představy kontrafaktuálních světů\footnote[13]{Alternativní svět za jiných okolností.}. To znamená, že superdeterministické teorie vylučují svobodnou vůli jako možnost jednat jinak. Z tohoto důvodu většina vědců tyto teorie odmítá. Odmítat vědeckou teorii jen kvůli tomu, že se vám nelíbí její důsledky, je nevědecké a kontraproduktivní.

Statistická Nezávislost neříká nic o Ko, takže Superdeterminismus tuto možnost nevylučuje.

\subsubsection{Svobodná vůle iluzí}
Z vědeckého pohledu se vše chová podle přírodních zákonů. Podle všech dosavadních experimentů vše následuje zákonitosti popsané s velmi dobrou přesností fyzikálními teoriemi. Tato skutečnost nás vede k vědeckému determinismu, který je neslučitelný se svobodnou vůlí. Podle determinismu jsou všechna naše rozhodnutí předurčena. Nemůžeme zvolit mezi několika možnými variantami budoucnosti, protože existuje pouze jedna varianta. Pravděpodobnost kvantové mechaniky nám také nedává svobodnou vůli. Podle kvantové mechaniky nemůžeme určit jak se rozhodneme, můžeme určit jen pravděpodobnost daného rozhodnutí. Ale pokud jsou naše rozhodnutí určena pravděpodobností, tak nejsou svobodná. Stejně jako moje rozhodnutí nebudou svobodná, když se budu rozhodovat pomocí hodu mince.

Ve skutečnosti se děje to, že náš mozek se rozhodne pomocí složitých nevědomých kalkulací a až po nějaké době se toto rozhodnutí dostane do našeho vědomí a projeví se jako iluze svobodného rozhodnutí. V roce 2008 byl proveden fascinující experiment \parencite{DecisionDet}, ve kterém byla použita funkční magnetická resonance k určení časového rozdílu mezi nevědomým a vědomým rozhodnutím. Účastnící experimentu si vybrali mezi dvěma tlačítky a okamžitě zmáčkli to, které si vybrali. Výzkumníci zjistili, že výsledek rozhodnutí je zakódován v mozkové aktivitě až 10 sekund předtím, než se dostane do vědomí. Teoreticky bychom tedy mohli předpovědět, jak se člověk rozhodne, když je ještě v procesu rozhodování.

V případě, že odmítneme determinismus tím, že předpokládáme existenci nějaké duše, která se rozhoduje nezávisle na fyzikálních zákonech, měli bychom pozorovat velké odchylky od našich fyzikálních teorií. Není tomu tak. Současné fyzikální teorie jsou podle velmi extenzivních testů (\cite{GenRelAcc}; \citetitle{QEDAcc}, \cite*{QEDAcc}) neuvěřitelně přesné. I kdybychom ignorovali všechna tato fakta, zůstává problém, že nerozhodujeme o tom, jakou duši dostaneme.

Kompatibilismus (Ko) je filozofie, kterou zastává většina dnešních filozofů \parencite{FWSur}. Jde o pokus zachování svobodné vůle při přijmutí determinismu. Ko svobodná vůle je sice kompatibilní s determinismem a Superdeterminismem, ale není doopravdy svobodná. Ko pouze redefinuje svobodné rozhodnutí na rozhodnutí, které je určeno tím, co chceme. Tedy pro kompatibilisty je svobodná vůle tím samým jako vůle, jelikož to, co chceme, je naší vůlí. Rozhodnutí určeno naší vůlí není svobodné. Nikdo neurčuje, co chce. Např. když bych se rozhodoval mezi čajem a kávou, tak bych se rozhodnul pro to, co chci víc. K tomu, abych se rozhodnul pro čaj, musel bych ho chtít víc jak kávu. I kdybych se rozhodnul pro kávu s vědomím, že \uv{chci} čaj, jen abych získal zpět svou svobodnou vůli, tak bych se pořád rozhodoval podle toho, co chci, rozhodnul bych se pro kafe, protože bych chtěl \uv{získat zpět} svou svobodnou vůli. Abychom změnili chtění na nechtění, museli bychom chtít nechtít a naopak. Je to pořád o tom, co chceme, jenomže my neurčujeme co chceme.

Abychom měli svobodnou vůli, museli bychom si být vědomi všeho, co nás ovlivňuje, a mít nad tím kontrolu, což není pravdou a je to samo o sobě paradox. Kdybychom měli tyto schopnosti, podle čeho bychom se rozhodovali? Jestliže mám kontrolu nad tím, co určuje, jak tuto kontrolu využiji, tak jsou moje volby určeny mými volbami, což je velice matoucí a nesmyslné.

Svobodná vůle je vlastně oxymorón. Když se rozhodnu podle své vůle, tak je mé rozhodnutí určeno mou vůlí (tím, co chci), takže nemůže být svobodné, a když by mé rozhodnutí nebylo určené, tak by se nejednalo o vůli.

Fakt, že Superdeterminismus vylučuje svobodnou vůli, není vůbec problematický, jelikož svobodná vůle je pouze iluzí a logicky nesouvislým nesmyslem. Ať je náš vesmír Superdeterministický, či ne, nemáme svobodnou vůli.

\clearpage

\subsection{Ohrožení vědecké metody}

Přesvědčeni o nezbytnosti předpokladu Statistické Nezávyslosti k objevování přírodních zákonů pomocí experimentů, \citeauthor{ArgSciMeth:1} už roku 1976 argumentovali proti Superdeterminismu:
\begin{quote}
    \emph{But, we maintain, skepticism of this sort will essentially dismiss all results of scientific experimentation. Unless we proceed under the assumption that hidden conspiracies of this sort do not occur, we have abondoned in advance the whole enterprise of discovering the laws of nature by experimentation.} - \cite{ArgSciMeth:1}
\end{quote}

Vědci se obávají, že randomizované kontrolované studie\footnote[14]{Osoby z experimentálního souboru jsou náhodně rozděleny do 2 stejně velkých souborů: pokusného (dostanou testovaný lék) a kontrolního (dostanou placebo).} by byly nemožné, kdyby výběr kontrolní skupiny mohl záviset na tom, co později změříme. Představme si, že náhodně rozdělíme lidi do dvou skupin k ověření účinnosti nového léku. Pokusná skupina dostane nový lék a kontrolní skupina placebo. V tomto případě je přiřazení do skupiny skrytá proměnná, $\lambda$. Když někdo onemocní, provedeme řadu testů (měření), abychom zjistili příčinu jeho nemoci. Pokud si myslíte, že to, co se stane s lidmi (jejich onemocnění), závisí na měření, které na nich provedete, nemůžete posoudit účinnost léku.

Tento argument vychází z představy, že z důvodu užitečnosti předpokladu Statistické Nezávislosti k pochopení vlastností klasických makroskopických systémů musí tento předpoklad platit i pro kvantové systémy. Tento závěr je zřetelně neoprávněný. Hlavní důvod této diskuze je nedostatečnost klasické fyziky k popisu kvantových systémů. A lidé se nechovají jako částice. Tento argument je ekvivalentní myšlence, že Schrödingerova kočka\footnote[15]{Schrödingerova kočka je myšlenkový experiment, ve kterém je po hodině, z důvodu kvantových účinků, kočka v krabici na 50\% mrtvá a na 50\% živá. A tím je podle kvantové mechaniky živá a mrtvá zároveň.} je doopravdy živá a mrtvá zároveň. V Superdeterminismu dochází k porušení Statistické Nezávislosti, pouze když kvantová mechanika předpovídá kolaps vlnové funkce, neboli při měření. Je důležité poznamenat, že Bellova nerovnost se zabývá pouze tím, co se děje při procesu měření. Ale jakmile změříme kvantový stav, tak tím porušování Statistické Nezávislosti končí. Měli bychom dodat, že měření nevyžaduje měřící přístroj. Měření je jakákoli dostatečně silná interakce s prostředím. Z tohoto důvodu v našem světě nepozorujeme živomrtvé kočky a superdeterministické korelace v lidech, protože vždy existuje nějaké prostředí (vzduch, světlo, nebo reliktní záření\footnote[16]{Elektromagnetické záření, které přichází z vesmíru ze všech směrů a je považováno za pozůstatek konce Velkého třesku.}).