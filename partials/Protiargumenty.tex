\section{Protiargumenty}
(Zpracováno podle článku \cite{supdet:rethink})
\subsection{Bellovy testy}


Nobelovu cenu za fyziku za rok 2022 dostala třetice fyziků Alain Aspect, Anton Zeilinger a John Clauser. Většina médií milně uvádí, že tato trojice dostala nobelovu cenu za dokázání kvantového provázání částic pomocí mnoha extenzivních testů Bellovy nerovnosti. Žádný z těchto experimentů nedokazuje kvantové provázání, ani nemožnost Superdeterminismu.

\subsubsection{Kosmické Bellovy testy}
V Kosmických Bellových testech \parencite{CosBTest:1}\parencite{CosBTest:2} jsou nastavení měření určena podle přesné vlnové délky světla, přicházejícího z kvazarů (velmi vzdálených objektů, které byly kauzálně odděleny v čas emise fotonů). Tento experiment je velmi pozoruhodný a je hoden nobelovy ceny, ale nemůže vyloučit Superdeterminismus. Pouze říká, že korelace, pozorovené v Bellových testech nemohly být lokálně způsobeny událostmi ve vzdálené minulosti (v případě Kosmických testů jsou těmito událostmi emise fotonů až miliardy let v minulosti, chvíli po vzniku vesmíru). Porušení Bellovy nerovnosti nám pouze říká, že jeden z předpokladů Bellovy nerovnosti byl porušen. Žádný Bellův test nemůže určit jaký předpoklad byl porušen.

Domněnka, že takové testy říkají něco o nemožnosti Superdeterminismu vycházejí z předpokladu, že stav blízký realizovanému stavu (např. kdyby světlo vyzařované vzdálenými kvazary mělo trošku jinou vlnovou délku) je povolen přírodními zákony a je pravděpodobný. V Superdeterministické teorii by ale tato malá změna vytvořila extrémně nepravděpodobný stav. V Superdeterministické teorii by změna vlnové délky světla z kvazarů mohla vyžadovat změnu jinde na hyperporvchu daného momentu, která by vedla k rozhodnutí experimentátora nepoužít světlo z daných kvazarů ve svém experimentu.
\clearpage

\subsubsection{Velký Bellův test}
Autoři Velkého Bellova testu \parencite{BigBTest} si uvědomili problém ve vytváření náhodnosti v nastavení měření k vyloučení korelace mezi skrytými proměnnými a nastavení měření (porušení Statistické Nezávislosti):
\begin{quote}
    A Bell test requires spatially distributed entanglement, fast and high-efficiency detection and unpredictable measurement settings. Although technology can satisfy the first two of these requirements the use of physical devices to choose settings in a Bell test involves making assumptions about the physics that one aims to test. Bell himself noted this weakness in using physical setting choices and argued that human ‘free will’ could be used rigorously to ensure unpredictability in Bell tests. Here we report a set of local-realism tests using human choices, which avoids assumptions about predictability in physics. 
\end{quote}

Velký Bellův test proběhl tak, že 100 000 lidských účastníků hrálo online videohru, čímž bylo vygenerováno 97 347 490 binárních voleb, které byly poslány do 12 laboratoří na pěti kontinentech, kde bylo provedeno 13 experimentů testujících Bellovu nerovnost pomocí fotonů, jednotlivých atomů, atomových souborů a supravodivých zařízení. Tímto experimentem byla údajně \uv{uzavřena mezera svobodné volby (možnost, že nastavení měření jsou ovlivněna skrytými proměnnými a tím korelována s vlastnostmi částic)}. Je jednoduché poznat problém v závěru, ke kterému dospěli autoři tohoto článku. Experimentátoři předpokládají, že vstup od lidí je na rozdíl od vstupu vytvářeného fyzickými přístroji náhodný a tím předpokládají fundamentální rozdíl mezi lidmi a fyzickými systémy (schopnost svobodné volby). Experiment předpokládá svobodu vůle účastníků při hraní videohry k vytvoření náhodnosti v nastavení měření a k následnému dokázání svobodné vůle. Svobodnou vůli nelze dokázat tím, že předpokládáme svobodnou vůli.

Ve skutečnosti moc nezáleží na detailech těchto experimentů. Musíme vzít na vědomí, že změření porušení Bellovy nerovnosti, ať je experiment jakkoliv komplexní, nemůže určit které předpoklady nerovnosti byly porušeny.


\subsection{Svobodná vůle}
Předpoklad Statistické Nezávislosti se většinou označuje jako předpoklad svobodné vůle, protože může být interpretován jako svoboda experimentátora vybrat si nastavení měření nezávisle na skrytých proměnných.

Je důležité rozlišovat dvě převládající definice svobodné vůle:
\begin{enumerate}
    \item \textbf{Libertariánská (Li):} možnost jednat jinak.
    \item \textbf{Kompatibilistická (Ko):} nepřítomnost omezení, která by člověku bránila dělat to, co si přeje.
\end{enumerate}

Statistická Nezávislost se opírá o Li, jelikož obojí vychází z představy kontrafaktuálních světů\footnote[13]{Alternativní svět za jiných okolností.}. To znamená, že Superdeterministické teorie vylučují svobodnou vůli jako možnost jednat jinak. Z tohoto důvodu většina vědců tyto teorie odmítá. Odmítat vědeckou teorii jen proto, že se vám nelíbí její důsledky, je nevědecké a kontraproduktivní.

Statistická nezávislost neříká nic o Ko, takže Superdeterminismus tuto možnost nevylučuje.

\subsubsection{Svobodná vůle iluzí}
Fakt, že Superdeterminismus vylučuje svobodnou vůli není vůbec problematický, jelikož svobodná vůle je pouze logicky nesouvislou iluzí. Ať je náš vesmír Superdeterministický, či ne, stejně nemáme svobodnou vůli.

Z vědeckého pohledu se vše chová podle přírodních zákonů. Podle všech dosavadních experimentů vše následuje zákonitosti, popsané s velmi dobrou přesností fyzikálními teoriemi. Tato skutečnost nás vede k vědeckému determinismu, který je neslučitelný se svobodnou vůlí. Podle determinismu jsou všechna naše rozhodnutí předurčena. Nemůžeme zvolit mezi několika možnými variantami budoucnosti, protože existuje pouze jedna varianta. Pravděpodobnost kvantové mechaniky nám také nedává svobodnou vůli. Podle kvantové mechaniky nemůžeme určit jak se rozhodneme, můžeme určit jen pravděpodobnost daného rozhodnutí. Ale pokud jsou naše rozhodnutí určena pravděpodobností, tak nejsou svobodná. Stejně jako moje rozhodnutí nebudou svobodná, když se budu rozhodovat pomocí hodu mince.

Ve skutečnosti se děje to, že náš mozek se rozhodne pomocí složitých nevědomých kalkulací a až po nějaké době se toto rozhodnutí dostane do našeho vědomí a projeví se jako iluze svobodného rozhodnutí. V roce 2008 byl proveden fascinující experiment \parencite{DecisionDet}, ve kterém byla použita funkční magnetická resonance k určení časového rozdílu mezi nevědomým a vědomým rozhodnutím. Účastnící experimentu si vybrali mezi dvěma tlačítky a okamžitě zmáčkli to, které si vybrali. Výzkumníci zjistili, že výsledek rozhodnutí je zakódován v mozkové aktivitě až 10 sekund předtím, než se dostane do vědomí. Teoreticky bychom tedy mohli předpovědět jak se člověk rozhodne, když je ještě v procesu rozhodování.

V případě, že odmítneme determinismus tím, že předpokládáme existenci nějaké duše, která se rozhoduje nezávisle na fyzikálních zákonech, měli bychom pozorovat velké odchylky od našich fyzikálních teorií. Není tomu tak. Současně fyzikální teorie jsou podle velmi extenzivních testů \parencite{GenRelAcc}\parencite{QEDAcc} neuvěřitelně přesné. I kdybychom ignorovali všechna tyto fakta, zůstává problém, že nerozhodujeme o tom jakou duši dostaneme.

Kompatibilismus je filozofie, kterou zastává většina dnešních filozofů \parencite{FWSur}. Jde o pokus zachování svobodné vůle při přijmutí determinismu. Ko svobodná vůle je sice kompatibilní s determinismem a Superdeterminismem, ale není doopravdy svobodná. Ko pouze redefinuje svobodné rozhodnutí na rozhodnutí, které je určeno tím co chci. Takže pro kompatibilisty je svobodná vůle to samé jako vůle, jelikož to co chci je má vůle. Rozhodnutí, určeno mou vůlí není svobodné. Nikdo neurčuje co chce. Když bych se rozhodoval mezi

\subsection{Ohrožení vědecké metody}