\section{Superdeterminismus}
\subsection{Definující vlastnosti}
(Zpracováno podle článku \cite{supdet:rethink})

Superdeterministické modely jsou Psi-epistemické, deterministické, lokální modely skrytých proměnných, které porušují Statistickou Nezávislost a nemusí být nutně realistické.

\subsubsection{Psi-epistemická}
Podle Psi-epistemické teorie vlnová funkce Schrödingerovy rovnice (Psi, $|\psi|$) neodpovídá přímo vlastnosti nějakého systému v reálném světě. Kodaňská interpretace Kvantové mechaniky je Psi-epistemická, protože považuje vlnovou funkci pouze jako reprezentaci znalostí o stavu systému.

Opakem je Psi-ontická teorie, která bere vlnovou funkci jako fundamentální část reálného světa.

Superdeterministické teorie jsou Psi-epistemické v tom smyslu, že vlnová funkce je průměrná pravděpodobnostní reprezentace přesných veličin systému, popsaných hlubší teorií.

Vlnová funkce odvozená ze superdeterministické teorie by se měla řídit dosud ověřenými evolučními zákony Kvantové mechaniky. Smysl hledání takové teorie je tedy vytváření předpovědí nad rámec Kvantové mechaniky.

\subsubsection{Deterministická}
Roku 1814 formuloval matematik Pierre-Simon de Laplace myšlenku deterministického vesmíru pomocí Laplaceova démona \parencite{laplace:demon}. Podle determinismu, by bytost (démon) znající pozici a momentum každé částice ve vesmíru a mající dostatečnou výpočetní sílu mohla pomocí fundamentálních zákonů přírody vypočítat minulost i budoucnost každé částice. Vše je předurčené, evoluce každé částice je dána přírodními zákony.

Determinismem myslíme, že evoluční zákon teorie jednoznačně mapuje stavy systému v čase \textbf{\emph{t}} na stavy v čase \textbf{\emph{t'}} pro libovolné \textbf{\emph{t}} a \textbf{\emph{t'}}.

Jelikož Kvantová mechanika není deterministická, musí deterministická teorie reprodukující Kvantovou mechaniku obsahovat skryté proměnné. Skryté proměnné, nadále kolektivně označované $\bm{\lambda}$, obsahují všechny informace potřebné k určení výsledku měření kromě \uv{neskrytých} proměnných, které jsou obsaženy v přípravě stavu systému.

Je důležité poznamenat, že tyto skryté proměnné nejsou vlastní pro, nebo lokalizované v měřeném systému. Představme si kluka jménem Nikolaj, který má na mysli dvě otázky: Jaká je moje hmotnost? a Zvládnu úspěšně udělat maturitní zkoušky? V deterministickém vesmíru se odpovědi na obě otázky nacházejí v současném stavu vesmíru, ale jejich dostupnost je velmi odlišná. Informace o hmotnosti Nikolaje se vyskytuje lokálně v něm samotném, zatímco informace o jeho úspěšnosti na maturitní zkoušce je rozložena po většině prostoru současné chvíle.

\subsubsection{Lokální} Lokalitou v Superdeterminismu exkluzivně myslíme Kontinuitu Působení (dále jen KoP). KoP znamená, že k přesunutí informace z jedné oblasti časoprostoru do jiné, oddělené oblasti, musí tyto informace být přítomné na jakémkoliv uzavřeném (3dimenzionálním) povrchu obklopujícím první oblast.


\begin{figure}[ht]

    \centering

    \begin{tikzpicture}
        % S
        \draw (2, 3) circle (0.8);
        \draw (2, 3) circle (0.65);
        
        %1
        \draw (2, 3) node  (1) {$1$};
        \draw (2, 3) circle (0.3);
        
        %2
        \draw (4, 3.3) node  (2) {$2$};
        \draw (4, 3.3) circle (0.3);
        
        
        % Axes
        \draw[->] (0, 0) -- (7,0) node[below] (x,y,z) {$x,y,z$};
        \draw[->] (0, 0) -- (0,7) node[left] (t) {$t$};
    \end{tikzpicture}
    \caption{\label{fig:7}Kontinuita Působení.}
\end{figure}


\subsubsection{Porušení Statistické Nezávislosti}
Korelace mezi propletenými částicemi v našem vesmíru porušují Bellovu nerovnost. Porušení této nerovnosti poukazuje na chybnost alespoň jednoho z předpokladů, potřebných k odvození Bellovy nerovnosti. Většinou je porušení Bellovy nerovnosti interpretováno jako důkaz nemožnosti lokálně realistické teorie \parencite{belltest:violation}. Tato interpretace nás nutí k výběru mezi lokalitou a realismem. K derivaci Bellovy nerovnosti je ale zapotřebí ještě předpoklad Statistické Nezávislosti.