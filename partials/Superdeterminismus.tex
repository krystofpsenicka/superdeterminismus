\section{Superdeterminismus}
\subsection{Definující vlastnosti}
(Zpracováno podle článku \cite{supdet:rethink})

Superdeterministické modely jsou Psi-epistemické, deterministické, lokální modely skrytých proměnných, které porušují Statistickou Nezávislost a nemusí být nutně realistické.

\subsubsection{Psi-epistemická}
Podle Psi-epistemické teorie vlnová funkce Schrödingerovy rovnice (Psi, $|\psi|$) neodpovídá přímo vlastnosti nějakého systému v reálném světě. Kodaňská interpretace Kvantové mechaniky je Psi-epistemická, protože považuje vlnovou funkci pouze jako reprezentaci znalostí o stavu systému.

Opakem je Psi-ontická teorie, která bere vlnovou funkci jako fundamentální část reálného světa.

Superdeterministické teorie jsou Psi-epistemické v tom smyslu, že vlnová funkce je průměrná pravděpodobnostní reprezentace přesných veličin systému, popsaných hlubší teorií.

Vlnová funkce odvozená ze superdeterministické teorie by se měla řídit dosud ověřenými evolučními zákony Kvantové mechaniky. Smysl hledání takové teorie je tedy vytváření předpovědí nad rámec Kvantové mechaniky.

\subsubsection{Deterministická}
Roku 1814 formuloval matematik Pierre-Simon de Laplace myšlenku deterministického vesmíru pomocí Laplaceova démona \parencite{laplace:demon}. Podle determinismu, by bytost (démon) znající pozici a momentum každé částice ve vesmíru a mající dostatečnou výpočetní sílu mohla pomocí fundamentálních zákonů přírody vypočítat minulost i budoucnost každé částice. Vše je předurčené, evoluce každé částice je dána přírodními zákony.

Determinismem myslíme, že evoluční zákon teorie jednoznačně mapuje stavy systému v čase \textbf{\emph{t}} na stavy v čase \textbf{\emph{t'}} pro libovolné \textbf{\emph{t}} a \textbf{\emph{t'}}.

Jelikož Kvantová mechanika není deterministická, musí deterministická teorie reprodukující Kvantovou mechaniku obsahovat skryté proměnné. Skryté proměnné, nadále kolektivně označované $\bm{\lambda}$, obsahují všechny informace potřebné k určení výsledku měření kromě \uv{neskrytých} proměnných, které jsou obsaženy v přípravě stavu systému.

Je důležité poznamenat, že tyto skryté proměnné nejsou vlastní pro, nebo lokalizované v měřeném systému. Představme si kluka jménem Nikolaj, který má na mysli dvě otázky: Jaká je moje hmotnost? a Zvládnu úspěšně udělat maturitní zkoušky? V deterministickém vesmíru se odpovědi na obě otázky nacházejí v současném stavu vesmíru, ale jejich dostupnost je velmi odlišná. Informace o hmotnosti Nikolaje se vyskytuje lokálně v něm samotném, zatímco informace o jeho úspěšnosti na maturitní zkoušce je rozložena po většině prostoru současné chvíle.

\subsubsection{Lokální}
(Zpracováno podle článku \cite{CoA})

Lokalitou v Superdeterminismu exkluzivně myslíme Kontinuitu Působení (dále jen KoP). Zvažme oddělené časoprostorové oblasti $\bm{1}$ a $\bm{2}$ (viz Obrázek \ref{fig:7}), přičemž $\bm{1}$ je obklopena \uv{zastiňovací} oblastí $\bm{S}$. $\bm{S}$ není pouze prostorovou oblastí, zahrnuje budoucnost i minulost oblasti $\bm{1}$ a zároveň i její prostorový rozsah (v dimenzích $\bm{x,y,z}$).

\begin{figure}[ht]

    \centering

    \begin{tikzpicture}[scale=0.85]
        % S
        \draw (2, 4) circle (1.3);
        \draw (2, 4) circle (1);

        \draw[<-] (3.15, 4) -- (4, 3) node[below] (S) {$S$};
        
        %1
        \draw (2, 4) node  (1) {$1$};
        \draw (2, 4) circle (0.5);
        
        %2
        \draw (5.5, 3) node  (2) {$2$};
        \draw (5.5, 3) circle (0.5);
        
        
        % Axes
        \draw[->] (0, 0) -- (7,0) node[below] (x,y,z) {$x,y,z$};
        \draw[->] (0, 0) -- (0,7) node[left] (t) {$t$};
    \end{tikzpicture}
    \caption{\label{fig:7}Kontinuita Působení zobrazená v časoprostorovém diagramu. $t$ je časová osa a $x,y,z$ je prostorová osa, znázorňující všechny 3 prostorové dimenze.}
\end{figure}

Matematický model porušuje KoP, pokud dovoluje \uv{působení na dálku}, tzn. pokud změny ve $\bm{2}$ souvisejí se změnami v $\bm{1}$, aniž by souvisely se změnami uvnitř $\bm{S}$. $\bm{S}$ je jakási kontrolovací oblast pro KoP. Jestliže se nějaká informace dostane z $\bm{1}$ do $\bm{2}$, musí se také nacházet v $\bm{S}$, aby model dodržoval KoP. Aby model s kohoutkem ve $\bm{2}$ a korelovanou fontánou v $\bm{1}$ splňoval KoP, musí obsahovat popis zprostředkujících parametrů (Např. tok vody trubkami mezi kohoutkem a fontánou) v přechodné zastiňovací oblasti $\bm{S}$. V takovém modelu jsou při znalosti všech parametrů v $\bm{S}$ dodatečné informace z $\bm{2}$ zbytečné k předpovědi budoucího vývoje $\bm{1}$.

Matematicky můžeme KoP vyjádřit rovnicí \ref{eq:7}. $\bm{I_{1}}$ a $\bm{I_{2}}$ představují množiny všech vstupů v oblastech $\bm{1}$ a $\bm{2}$ postupně. $\bm{Q_{1}}$, $\bm{Q_{2}}$ a $\bm{Q_{S}}$ označují nevstupní parametry v odpovídající oblasti.

\begin{equation}
    \bm{P_{I_{1},I_{2}}(Q_{1}|Q_{2}, Q_{S}) = P_{I_{1}}(Q_{1}|Q_{S})}
    \label{eq:7}
\end{equation}

Tato rovnice vyjadřuje nezávislost evolučního zákona $\bm{P_{I_{1}}(Q_{1}|Q_{S})}$ na vstupech $\bm{I_{2}}$ a parametrech $\bm{Q_{2}}$. Jinými slovy pravděpodobnostní distribuce parametrů $\bm{Q_{1}}$ se vstupy $\bm{I_{1}}$ a $\bm{I_{2}}$ za předpokladu znalosti $\bm{Q_{2}}$ a $\bm{Q_{S}}$ je stejná jako ta samá pravděpodobnostní distribuce bez vstupů $\bm{I_{2}}$ a parametrů $\bm{Q_{2}}$. Když je tato podmínka splněna, říkáme, že $\bm{S}$ zastiňuje $\bm{1}$ od $\bm{2}$. U modelů splňujících KoP musí tato rovnost platit pro všechny jednoduše propojené, nepřekrývající se oblasti $\bm{1}$, $\bm{2}$ a $\bm{S}$, pro které platí, že oblast $\bm{S}$ zcela odděluje $\bm{1}$ od $\bm{2}$ a nikde není mizivě tenká.

Bellova Lokalita (dále jen BL), použitá k odvození Bellovy nerovnosti, je silnější kritérium než KoP. BL má oproti KoP ještě 2 omezení:

\begin{enumerate}
    \item \textbf{Nezávislost na Budoucím Vstupu}
     
    Nezávislost na budoucím vstupu (dále jen NBV) zmenšuje zastiňovací oblast na část $\bm{S'}$, která neleží v budoucnosti obou oblastí $\bm{1}$ a $\bm{2}$ (viz Obrázek \ref{fig:8}).

    NBV platí pro matematický model $\bm{P_{I}(Q)}$, jestliže existuje model $\bm{P'_{I'}(Q')}$ omezený časem $\bm{t'}$\footnote[5]{Horní časová hranice časoprostorových oblastí $\bm{1}$ a $\bm{2}$.}, který splňuje rovnici \ref{eq:8}. $\bm{I'}$ je množina všech vstupů v časech po $\bm{t'}$  a $\bm{Q'}$ je množina všech nevstupových parametrů v časech po $\bm{t'}$.

    NBV říká, že $\bm{P_{I}(Q')}$ je nezávislý na budoucích vstupech.

    \begin{equation}
        \bm{P_{I}(Q') = P'_{I'}(Q')}
        \label{eq:8}
    \end{equation}

    \clearpage
    
    \begin{figure}[h]

        \centering
    
        \begin{tikzpicture}[scale=0.85]
            % S
            \draw (2, 4) circle (1.3);
            \draw (2, 4) circle (1);
    
            \draw[<-] (3.15, 4) -- (4, 3) node[below] (S') {$S'$};
            
            %1
            \draw (2, 4) node  (1) {$1$};
            \draw (2, 4) circle (0.5);
            
            %2
            \draw (5.5, 3) node  (2) {$2$};
            \draw (5.5, 3) circle (0.5);
            
            %hide future
            \fill [white] (0.1,4.5) rectangle (6,6);
            \draw[line width=0.5mm,dotted] (0.3,4.5) -- (4, 4.5);
    
            % Axes
            \draw[->] (0, 0) -- (7,0) node[below] (x,y,z) {$x,y,z$};
            \draw[->] (0, 0) -- (0,7) node[left] (t) {$t$};
        \end{tikzpicture}
        \caption{\label{fig:8}Nezávislost na budoucím vstupu. $\bm{S'}$ je oblast $\bm{S}$ omezena na minulost a přítomnost oblastí $\bm{1}$ a $\bm{2}$.}
    \end{figure}
    

    \item \textbf{Platnost zastiňovací oblasti pro všechny referenční rámce (pozorovatele).}

    K pochopení tohoto omezení si nejdříve vysvětlíme časoprostorové diagramy, světelné kužely a referenční rámce.

    Na obrázku \ref{fig:9} vidíme časoprostorový diagram. Osa $\bm{ct}$ je osa času (vynásobeného rychlostí světla $\bm{c}$) a osa $\bm{x,y,z}$ je osa prostoru, představující všechny 3 prostorové dimenze. V počátku soustavy souřadnic je nějaká událost $\bm{U}$. 
    
    Dráha, kterou objekt sleduje v časoprostorovém diagramu se nazývá světočára. Světočára elektronu je vždy přímka s úhlem 45° od osy $\bm{x,y,z}$, jelikož elektron má rychlost světla. Takže na této soustavě souřadnic představuje světočáru elektronu rovnice $\bm{ct=(x,y,z)}$, nebo také $\bm{ct=-(x,y,z)}$, která říká, že dráha, kterou elektron ucestuje prostorem je rovna produktu rychlosti světla a času, který uběhne.

    Když do diagramu nakreslíme světočáry elektronu, vzniknou dva světelné kužely\footnote[6]{Kužely, protože ve skutečnosti jsou ve 4 dimenzích.}. Podle speciální teorie relativity\parencite{SpRel} se nemůže kauzální vliv\footnote[7]{Jakákoliv informace(vliv, síla).} šířit rychleji než světlo. Budoucí světelný kužel tedy obsahuje všechny události, které může událost $\bm{U}$ kauzálně ovlivnit. A minulý světelný kužel obsahuje všechny události, které mohly ovlivnit událost $\bm{U}$.

    \clearpage

    \begin{figure}[h]

        \centering
    
        \begin{tikzpicture}[scale=1.7]
           
            \def\xmax{2}
            \def\xmaxp{2.2} % maximum of rotated axis
            \def\Nlines{5} % number of world lines (at constant x/t)
            \pgfmathsetmacro\d{0.9*\xmax/\Nlines} % grid size
            \pgfmathsetmacro\ang{atan(1/3)} % angle between x and x' axes
            \coordinate (O) at (0,0);
            \coordinate (X) at (\xmax+0.2,0);
            \coordinate (T) at (0,\xmax+0.2);
            \coordinate (C) at (45:\xmaxp+0.2);
            \coordinate (E) at (4*\d,0); % event
            
            % WORLD LINE GRID
            \foreach \i [evaluate={\x=\i*\d;}] in {1,...,\Nlines}{
              \message{  Running i/N=\i/\Nlines, x=\x...^^J}
              \draw[world line]   (-\x,-\xmax) -- (-\x,\xmax);
              \draw[world line]   ( \x,-\xmax) -- ( \x,\xmax);
              \draw[world line t] (-\xmax,-\x) -- (\xmax,-\x);
              \draw[world line t] (-\xmax, \x) -- (\xmax, \x);
            }
            
            % AXES
            \draw[->,thick] (0,-\xmax) -- (T) node[left] {$ct$};
            \draw[->,thick] (-\xmax,0) -- (X) node[below] {$x,y,z$};
            
            % LABELS
            \node[black,above] at (0,1) {budoucí světelný kužel};
            \node[black,below] at (0,-1) {minulý světelný kužel};

            %fill cones
            \fill[black,opacity=0.15] % TIMELIKE
    (\xmax,\xmax) -- (-\xmax,\xmax) -- (\xmax,-\xmax) -- (-\xmax,-\xmax) -- cycle;

            % PHOTON
  \draw[photon] ( \xmax,-\xmax) -- ( 0.02*\xmax,-0.02*\xmax);
  \draw[photon] (-\xmax,-\xmax) -- (-0.02*\xmax,-0.02*\xmax);
  \draw[photon] ( 0.02*\xmax,0.02*\xmax) -- ( \xmax,\xmax);
  \draw[photon] (-0.02*\xmax,0.02*\xmax) -- (-\xmax,\xmax);
            
            %event
            \fill[black] (O) circle(0.1);
            \node[white] at (0,0) {\scriptsize $U$};

        \end{tikzpicture}
        \caption{\label{fig:9}Světelné kužely události $\bm{U}$ v časoprostorovém diagramu.}
    \end{figure}

Na obrázku \ref{fig:10} je zobrazena časoprostorová soustava souřadnic pozorovatele $\bm{A}$, který se vzhledem k události $\bm{U}$ pohybuje rychlostí $0$ a přes ní je zobrazena časoprostorová soustava pozorovatele $\bm{B}$, který se vzhledem k události $\bm{U}$ pohybuje rychlostí $\bm{0.3c}$ ($\bm{30\%}$ rychlosti světla). Z diagramu můžeme vidět, že události $\bm{T}$,$\bm{U}$,$\bm{V}$ probíhají současně pro pozorovatele $\bm{A}$, ale pro pozorovatele $\bm{B}$ probíhají v pořadí $\bm{V}$,$\bm{U}$,$\bm{T}$. Světelné kužely události $\bm{U}$ zůstávají stejné pro všechny pozorovatele, jelikož světočára fotonu je pořád stejná: rovnice $\bm{ct=\pm(x,y,z)}$ pro pozorovatele $\bm{A}$ a $\bm{ct'=\pm(x',y',z')}$ pro pozorovatele $\bm{B}$ vykreslují stejné přímky (hranice světelných kuželů události $\bm{U}$). Takže omezení kauzálních vlivů světelnými kužely platí pro všechny pozorovatele.

\begin{figure}[h]

    \centering

    \begin{tikzpicture}[scale=1.8]
       
        \def\xmax{2}
        \def\xmaxp{2.2} % maximum of rotated axis
        \def\Nlines{5} % number of world lines (at constant x/t)
        \pgfmathsetmacro\d{0.9*\xmax/\Nlines} % grid size
        \pgfmathsetmacro\ang{atan(1/3)} % angle between x and x' axes
        \pgfmathsetmacro\D{\d/cos(\ang)/sqrt(1-tan(\ang)^2)} % grid size, boosted
        \coordinate (O) at (0,0);
        \coordinate (X) at (\xmax+0.2,0);
        \coordinate (T) at (0,\xmax+0.2);
        \coordinate (C) at (45:\xmaxp+0.2);
        \coordinate (E) at (4*\d,0); % event
        \coordinate (X') at (\ang:\xmaxp+0.2);
        \coordinate (T') at (90-\ang:\xmaxp+0.2);
        
        % WORLD LINE GRID
        \foreach \i [evaluate={\x=\i*\d;}] in {1,...,\Nlines}{
          \message{  Running i/N=\i/\Nlines, x=\x...^^J}
          \draw[world line, opacity=0.5]   (-\x,-\xmax) -- (-\x,\xmax);
          \draw[world line, opacity=0.5]   ( \x,-\xmax) -- ( \x,\xmax);
          \draw[world line t, opacity=0.5] (-\xmax,-\x) -- (\xmax,-\x);
          \draw[world line t, opacity=0.5] (-\xmax, \x) -- (\xmax, \x);
        }
        
        % BOOSTED WORLD LINE GRID

        \foreach \i [evaluate={\x=\i*\D;}] in {1,...,\Nlines}{
            \message{  Running i/N=\i/\Nlines, x=\x...^^J}
            \draw[world line', opacity=0.7] (\ang:-\x) --++ (90-\ang:-\xmaxp);
            \draw[world line', opacity=0.7] (90-\ang:-\x) --++ (\ang:-\xmaxp);
            \draw[world line', opacity=0.7] (\ang:\x) --++ (90-\ang:\xmaxp);
            \draw[world line', opacity=0.7] (90-\ang:\x) --++ (\ang:\xmaxp);
            \draw[world line', opacity=0.7] (\ang:-\x) --++ (90-\ang:\xmaxp);
            \draw[world line', opacity=0.7] (90-\ang:-\x) --++ (\ang:\xmaxp);
            \draw[world line', opacity=0.7] (\ang:\x) --++ (90-\ang:-\xmaxp);
            \draw[world line', opacity=0.7] (90-\ang:\x) --++ (\ang:-\xmaxp);
        }
        
        % AXES
        \draw[->,thick] (0,-\xmax) -- (T) node[left] {$ct$};
        \draw[->,thick] (-\xmax,0) -- (X) node[below] {$x,y,z$};

        %boosted axes
        \draw[->,thick] (90-\ang:-\xmaxp) -- (T') node[left] {$ct'$};
        \draw[->,thick] (\ang:-\xmaxp) -- (X') node[below] {$x',y',z'$};
        
        %fill cones
        \fill[black,opacity=0.15] % TIMELIKE
(\xmax,\xmax) -- (-\xmax,\xmax) -- (\xmax,-\xmax) -- (-\xmax,-\xmax) -- cycle;

        % PHOTON
\draw[photon] ( \xmax,-\xmax) -- ( 0.02*\xmax,-0.02*\xmax);
\draw[photon] (-\xmax,-\xmax) -- (-0.02*\xmax,-0.02*\xmax);
\draw[photon] ( 0.02*\xmax,0.02*\xmax) -- ( \xmax,\xmax);
\draw[photon] (-0.02*\xmax,0.02*\xmax) -- (-\xmax,\xmax);
        
        %events
        \fill[black] (O) circle(0.1);
        \node[white] at (0,0) {\scriptsize $U$};
        \fill[black] (-1, 0) circle(0.1);
        \node[white] at (-1,0) {\scriptsize $T$};
        \fill[black] (1,0) circle(0.1);
        \node[white] at (1,0) {\scriptsize $V$};

    \end{tikzpicture}
    \caption{\label{fig:10}Posloupnost událostí $\bm{T}$,$\bm{U}$,$\bm{V}$ pro 2 různé pozorovatele.}
\end{figure}

\clearpage

Aby tedy lokální kauzalita platila pro všechny pozorovatele, musíme zastiňovací oblast omezit světelnými kužely oblastí $\bm{1}$ a $\bm{2}$\footnote[8]{Zastiňovací oblast nestačí omezit světelnými kužely oblasti $\bm{1}$. Zastiňovací oblast musí zastiňovat oblast $\bm{1}$ od překryvu světelných kuželů oblastí $\bm{1}$ a $\bm{2}$.} (viz Obrázek \ref{fig:11}).



    \begin{figure}[h]

        \centering
    
        \begin{tikzpicture}[scale=0.9]
           %S
           
           \draw (2, 4) circle (1.3);
           \draw (2, 4) circle (1);
            
            %1
            \draw (2, 4) node  (1) {$1$};
            \draw (2, 4) circle (0.5);
            
            %2
            \draw (5.5, 3) node  (2) {$2$};
            \draw (5.5, 3) circle (0.5);
            
            %constrict to past lightcone
            \node[isosceles triangle,
    isosceles triangle apex angle=90,
    draw,
    anchor=apex,
    draw=none,
    fill=white,
    minimum size =1.5cm] (T90)at (1.67,4.37){};
            \node[isosceles triangle,
    isosceles triangle apex angle=90,
    draw,
    anchor=apex,
    draw=none,
    fill=white,
    rotate=180,
    minimum size =1.5cm] (T90)at (2.33,4.37){};
    \fill [white] (0.1,4.5) rectangle (6,6);

            \draw[line width=0.5mm,dotted] (-1.5,1.2) -- (1.67, 4.37);
            \draw[line width=0.5mm,dotted] (5.5,1.2) -- (2.33,4.37);
            
            %lightcone of 2
            \draw[line width=0.5mm,dotted] (5.17, 3.37) -- (2, 0.2);
            \draw[line width=0.5mm,dotted] (5.83, 3.37) -- (9,0.2);
             % S arrow
     
             \draw[<-] (2.65, 3.1) -- (3.7, 2.5) node[below] (S'') {$S''$};


            % Axes
            \draw[->] (0, 0) -- (7,0) node[below] (x,y,z) {$x,y,z$};
            \draw[->] (0, 0) -- (0,7) node[left] (t) {$t$};
        \end{tikzpicture}
        \caption{\label{fig:11}Bellova Lokalita. Splňující podmínky Nezávislosti na Budoucím Vstupu a Platnosti zastiňovací oblasti pro všechny referenční rámce.}
    \end{figure}
\end{enumerate}

\subsubsection{Porušení Statistické Nezávislosti}
Korelace mezi propletenými částicemi v našem vesmíru porušují Bellovu nerovnost. Porušení této nerovnosti poukazuje na chybnost alespoň jednoho z předpokladů, potřebných k odvození Bellovy nerovnosti. Většinou je porušení Bellovy nerovnosti interpretováno jako důkaz nemožnosti lokálně realistické teorie \parencite{belltest:violation}. Tato interpretace nás nutí k výběru mezi lokalitou a realismem. K derivaci Bellovy nerovnosti je ale zapotřebí ještě předpoklad Statistické Nezávislosti.