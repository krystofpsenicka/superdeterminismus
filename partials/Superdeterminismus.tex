\section{Superdeterminismus}
\subsection{Definující vlastnosti}
(Zpracováno podle článku \cite{supdet:rethink})

Superdeterministické modely jsou Psi-epistemické, deterministické, lokální modely skrytých proměnných, které porušují Statistickou Nezávislost a nemusí být nutně realistické.

\subsubsection{Psi-epistemická}

\subsubsection{Deterministická}

\subsubsection{Lokální}

\subsubsection{Porušení Statistické Nezávislosti}
Korelace mezi propletenými částicemi v našem vesmíru porušují Bellovu nerovnost. Porušení této nerovnosti poukazuje na chybnost alespoň jednoho z předpokladů, potřebných k odvození Bellovy nerovnosti. Většinou je porušení Bellovy nerovnosti interpretováno jako důkaz nemožnosti lokálně realistické teorie \parencite{belltest:violation}. Tato interpretace nás nutí k výběru mezi lokalitou a realismem. K derivaci Bellovy nerovnosti je ale zapotřebí ještě předpoklad Statistické Nezávislosti.