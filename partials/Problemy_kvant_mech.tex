\section{Problémy kvantové mechaniky}
    \subsection{Problém měření}
Kodaňská interpretace kvantové mechaniky má tři části: Schrödingerovu vlnovou rovnici, postulát o měření a Bornovo pravidlo.

Schrödingerova vlnová rovnice popisuje změnu vlnové funkce v čase kromě procesu měření. Tento proces chápeme jako interakci vlnové funkce měřeného systému s jiným (měřícím) systé\-mem, která zanechá informaci o velikosti určité veličiny měřeného systému v měřícím systému. Když dojde k takové interakci, musíme použít postulát o měření aby stav naší vlnové funkce souhlasil s realitou.

Postulát o měření můžeme jednoduše vysvětlit pomocí jednotkové kružnice. V úvodu jsem představil koncept vlnové funkce, používaný k vyjádření stavu kvantového systému. Tuto funkci můžeme, v případě kdy sledujeme binární veličinu kvantového systému, vyjádřit pomocí vektoru na jednotkové kružnici. 



\begin{figure}[h]

    \centering

    \begin{tikzpicture}
        % Define radius
        \def\r{2}
        % Define x of vector
        \def\x{1.2}
        % Define y of vector
        \def\y{1.6}

        % vector
        \draw (0,0) node[circle,fill,inner sep=1] (orig) {} -- (\x,\y) node[circle,fill,inner sep=0.7,label=above:$|\psi\rangle$] (a) {};
        \draw [dashed] (\x, 0) node[below] (x0) {$x_0$} -- (\x, \y) node[] (a) {};
        \draw [dashed] (0, \y) node[left] (y0) {$y_0$} -- (\x, \y) node[] (a) {};
        
        
        % Sphere
        \draw (orig) circle (\r);
        
        % Axes
        \draw[->] (-3, 0) -- (3,0) node[right] (x) {$x$};
        \draw[->] (0, -3) -- (0,3) node[above] (y) {$y$};
    \end{tikzpicture}
    \caption{\label{fig:3}}
\end{figure}