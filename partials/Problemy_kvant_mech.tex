\section{Problémy kvantové mechaniky}
    \subsection{Problém měření}
Kodaňská interpretace kvantové mechaniky má tři části: Schrödingerovu vlnovou rovnici, postulát o měření a Bornovo pravidlo.

Schrödingerova vlnová rovnice popisuje každou změnu vlnové funkce v čase (kromě procesu měření). Tento proces chápeme jako interakci vlnové funkce měřeného systému s jiným (měřícím) systé\-mem, která v měřící systému zanechá informaci o velikosti určité veličiny měřeného systému. Když dojde k takové interakci, musíme použít postulát o měření aby stav naší vlnové funkce souhlasil s realitou.

Postulát o měření můžeme jednoduše vysvětlit pomocí jednotkové kružnice. V úvodu jsem představil koncept vlnové funkce, používaný k vyjádření stavu kvantového systému. Když sledujeme binární veličinu kvantového systému (např. spin kvantové částice), můžeme využít znázorně\-ní, používaného v oboru Kvantového počítání k ilustraci stavu kvantového bitu (qubit). Stav qubitu se znázorňuje jako vektor $\bm{|\psi\rangle}$ na jednotkové kružnici v soustavě souřadnic, kde osa x je jeden stav qubitu (např. spin nahoru, označovaný podobně jako u bitů jako $\bm{|1\rangle}$) a osa y je stav druhý (např. spin dolů, jako $\bm{|0\rangle}$) (viz Obrázek \ref{fig:3}). Tento vektor $\bm{|\psi\rangle}$ zapisujeme jako součet možných výsledků (viz Rovnice (\ref{eq:3})), přičemž druhé mocniny koeficientů $\bm{\alpha}$ a $\bm{\beta}$ určují pravděpodobnost daného výsledku. 

\begin{equation}
    \bm{|\psi\rangle = \alpha|1\rangle + \beta|0\rangle}
    \label{eq:3}
\end{equation}

Součet pravděpodobností je vždy 1, takže pro koeficienty platí vztah $\bm{\alpha^2 + \beta^2 = 1}$. Takto se také počítá absolutní hodnota vektoru $\bm{|\psi\rangle}$, která je tedy vždy 1. To znamená, že délka vektoru $\bm{|\psi\rangle}$ je vždy 1, a proto používáme jednotkovou kružnici. 

\begin{figure}[ht]

    \centering

    \begin{tikzpicture}
        % Define radius
        \def\r{2}
        % Define x of vector
        \def\x{1.2}
        % Define y of vector
        \def\y{1.6}

        % vector
        \draw[->,line width=0.6mm, draw=red] (0,0) node[circle,fill,inner sep=1] (orig) {} -- (\x,\y) node[circle,fill,inner sep=0.7,label=above:$|\psi\rangle$] (a) {};
        \draw [dashed] (\x, 0) node[below] (A) {$\alpha$} -- (\x, \y) node[] (a) {};
        \draw [dashed] (0, \y) node[left] (B) {$\beta$} -- (\x, \y) node[] (a) {};
        
        
        % Sphere
        \draw (orig) circle (\r);
        
        % Axes
        \draw[->] (-3, 0) -- (3,0) node[right] (|1>) {$|1\rangle$};
        \draw[->] (0, -3) -- (0,3) node[above] (|0>) {$|0\rangle$};
    \end{tikzpicture}
    \caption{\label{fig:3}Znázornění vlnové funkce pomocí vektoru na jednotkové kružnici.}
\end{figure}

V praxi se používá Blochova sféra (jednotková sféra), jelikož koeficienty $\bm{\alpha}$ a $\bm{\beta}$ jsou komplexní čísla, která mají navíc imaginární rozměr, což znamená, že $\bm{|\psi\rangle}$ je trojrozměrný vektor. Nám k ilustraci postulátu o měření postačí 2 rozměry.

Podle postulátu o měření máme při měření vektor $\bm{|\psi\rangle}$ aktualizovat promítnutím na osu změře\-ného výsledku a následně ho prodloužit zpět na délku 1.

Např. když budeme měřit stav qubitu a změříme ho ve stavu $\bm{|1\rangle}$, musíme vektor aktualizovat, aby správně popisoval reálný stav qubitu. V tomto případě musíme vektor promítnout na osu změřeného stavu (osu $\bm{|1\rangle}$) viz Obrázek \ref{fig:4}. Nakonec musíme aktualizovat pravděpodobnost prodloužením vektoru zpět na délku 1 viz Obrázek \ref{fig:5}.

\begin{figure}[ht]

    \centering

    \begin{tikzpicture}
        % Define radius
        \def\r{2}
        % Define x of vector
        \def\x{1.2}
        % Define y of vector
        \def\y{1.6}

        % vectors
        % shadow vector
        \draw[->, draw=pink] (0,0) node[circle,fill,inner sep=1] (orig) {} -- (\x,\y) node[circle,fill,inner sep=0.7,label=above:$|\psi\rangle$] (a) {};
        % psi vector
        \draw[->,line width=0.6mm, draw=red] (0,0) node[circle,fill,inner sep=1] (orig) {} -- (\x,0) node[circle,fill,inner sep=0.7,label=above:$|\psi\rangle$] (a) {};
        \draw [dashed] (\x, 0) node[below] (A) {$\alpha$} -- (\x, \y) node[] (a) {};
        \draw [dashed] (0, \y) node[left] (B) {$\beta$} -- (\x, \y) node[] (a) {};
        
        
        % Sphere
        \draw (orig) circle (\r);
        
        % Axes
        \draw[->] (-3, 0) -- (3,0) node[right] (|1>) {$|1\rangle$};
        \draw[->] (0, -3) -- (0,3) node[above] (|0>) {$|0\rangle$};
    \end{tikzpicture}
    \caption{\label{fig:4}Promítnutí vektoru na osu změřeného výsledku.}
\end{figure}

\begin{figure}[ht]

    \centering

    \begin{tikzpicture}
        % Define radius
        \def\r{2}
        % Define x of vector
        \def\x{1.2}
        % Define y of vector
        \def\y{1.6}

        % vectors
        % shadow vector
        \draw[->, draw=pink] (0,0) node[circle,fill,inner sep=1] (orig) {} -- (\x,0) node[circle,fill,inner sep=0.7,label=above:$|\psi\rangle$] (a) {};
        % psi vector
        \draw[->,line width=0.6mm, draw=red] (0,0) node[circle,fill,inner sep=1] (orig) {} -- (\r,0) node[circle,fill,inner sep=0.7,label=above:$|\psi\rangle$] (a) {};
        \draw [dashed] (\x, 0) node[below] (A) {$\alpha$} -- (\x, \y) node[] (a) {};
        \draw [dashed] (0, \y) node[left] (B) {$\beta$} -- (\x, \y) node[] (a) {};
        
        
        % Sphere
        \draw (orig) circle (\r);
        
        % Axes
        \draw[->] (-3, 0) -- (3,0) node[right] (|1>) {$|1\rangle$};
        \draw[->] (0, -3) -- (0,3) node[above] (|0>) {$|0\rangle$};
    \end{tikzpicture}
    \caption{\label{fig:5}Prodloužení vektoru zpět na délku 1.}
\end{figure}

Problém měření spočívá v tom, že Schrödingerova rovnice je lineární a jak víme, nezahrnuje proces měření, který s ní není kompatibilní, protože postulát o měření není jednotková operace, protože je nevratná, takže není lineární. Konkrétněji, když připravíme částici ve stavu $|\psi\rangle = |1\rangle$, změříme ji jako $|1\rangle$ a když ji připravíme ve stavu $|\psi\rangle = |0\rangle$, změříme ji jako $|1\rangle$. Podle lineárního evolučního zákonu bychom změřili částici připravenou v superpozici obou stavů $|\psi\rangle = \sqrt{\frac{1}{2}} |0\rangle + \sqrt{\frac{1}{2}} |1\rangle$ jako superpozici obou stavů, ale v realitě tuto částici změříme jako $|1\rangle$, nebo $|0\rangle$.

Podle postulátu o měření se nástoje na měření nechovají podle Schrödingerovy rovnice. Aby kvatová mechanika tedy dávala smysl museli bychom opustit předpoklad redukcionizmu. Podle redukcionizmu se dá derivovat chování složitých systémů mnoha částic podle chování jejich konstituentů. Kdybychom opustili tento předpoklad museli bychom definovat hranici na které se redukcionizmus rozpadá.



