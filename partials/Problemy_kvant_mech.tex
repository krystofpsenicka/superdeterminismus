\section{Problémy kvantové mechaniky}
    \subsection{Problém měření}
Kodaňská interpretace kvantové mechaniky má tři části: Schrödingerovu vlnovou rovnici, postulát o měření a Bornovo pravidlo.

Schrödingerova vlnová rovnice popisuje každou změnu vlnové funkce v čase (kromě procesu měření). Tento proces chápeme jako interakci vlnové funkce měřeného systé\-mu s jiným (měřícím) systémem, která v měřící systému zanechá informaci o velikosti určité veličiny měřeného systému. Když dojde k takové interakci, musíme použít postulát o měření aby stav naší vlnové funkce souhlasil s realitou.

Postulát o měření můžeme jednoduše vysvětlit pomocí jednotkové kružnice. V úvodu jsem představil koncept vlnové funkce, používaný k vyjádření stavu kvantového systému. Když sledujeme binární veličinu kvantového systému (např. spin kvantové částice), můžeme využít znázorně\-ní, používaného v oboru Kvantového počítání k ilustraci stavu kvantového bitu (qubit). Stav qubitu se znázorňuje jako vektor $\bm{|\psi\rangle}$ na jednotkové kružnici v soustavě souřadnic, kde osa x je jeden stav qubitu (např. spin nahoru, označovaný podobně jako u bitů jako $\bm{|1\rangle}$) a osa y je stav druhý (např. spin dolů, jako $\bm{|0\rangle}$) (viz Obrázek \ref{fig:3}). Tento vektor $\bm{|\psi\rangle}$ zapisujeme jako součet možných výsledků (viz Rovnice (\ref{eq:3})), přičemž druhé mocniny koeficientů $\bm{\alpha}$ a $\bm{\beta}$ určují pravděpodobnost daného výsledku. 

\begin{equation}
    \bm{|\psi\rangle = \alpha|1\rangle + \beta|0\rangle}
    \label{eq:3}
\end{equation}

Součet pravděpodobností je vždy 1, takže pro koeficienty platí vztah $\bm{\alpha^2 + \beta^2 = 1}$. Takto se také počítá absolutní hodnota vektoru $\bm{|\psi\rangle}$, která je tedy vždy 1. To znamená, že délka vektoru $\bm{|\psi\rangle}$ je vždy 1, a proto používáme jednotkovou kružnici. 

\begin{figure}[ht]

    \centering

    \begin{tikzpicture}
        % Define radius
        \def\r{2}
        % Define x of vector
        \def\x{1.2}
        % Define y of vector
        \def\y{1.6}

        % vector
        \draw[->,line width=0.6mm, draw=red] (0,0) node[circle,fill,inner sep=1] (orig) {} -- (\x,\y) node[circle,fill,inner sep=0.7,label=above:$|\psi\rangle$] (a) {};
        \draw [dashed] (\x, 0) node[below] (A) {$\alpha$} -- (\x, \y) node[] (a) {};
        \draw [dashed] (0, \y) node[left] (B) {$\beta$} -- (\x, \y) node[] (a) {};
        
        
        % Sphere
        \draw (orig) circle (\r);
        
        % Axes
        \draw[->] (-3, 0) -- (3,0) node[right] (|1>) {$|1\rangle$};
        \draw[->] (0, -3) -- (0,3) node[above] (|0>) {$|0\rangle$};
    \end{tikzpicture}
    \caption{\label{fig:3}Znázornění vlnové funkce pomocí vektoru na jednotkové kružnici.}
\end{figure}

V praxi se používá Blochova sféra (jednotková sféra), jelikož koeficienty $\bm{\alpha}$ a $\bm{\beta}$ jsou komplexní čísla, která mají navíc imaginární rozměr, což znamená, že $\bm{|\psi\rangle}$ je trojrozměrný vektor. Nám k ilustraci postulátu o měření postačí 2 rozměry.

Podle postulátu o měření máme při měření vektor $\bm{|\psi\rangle}$ aktualizovat promítnutím na osu změře\-ného výsledku a následně ho prodloužit zpět na délku 1.

Např. když budeme měřit stav qubitu a změříme ho ve stavu $\bm{|1\rangle}$, musíme vektor aktualizovat, aby správně popisoval reálný stav qubitu. V tomto případě musíme vektor promítnout na osu změřeného stavu (osu $\bm{|1\rangle}$) viz Obrázek \ref{fig:4}. Nakonec musíme aktualizovat pravděpodobnost prodloužením vektoru zpět na délku 1 viz Obrázek \ref{fig:5}.

\begin{figure}[ht]

    \centering

    \begin{tikzpicture}
        % Define radius
        \def\r{2}
        % Define x of vector
        \def\x{1.2}
        % Define y of vector
        \def\y{1.6}

        % vectors
        % shadow vector
        \draw[->, draw=pink] (0,0) node[circle,fill,inner sep=1] (orig) {} -- (\x,\y) node[circle,fill,inner sep=0.7,label=above:$|\psi\rangle$] (a) {};
        % psi vector
        \draw[->,line width=0.6mm, draw=red] (0,0) node[circle,fill,inner sep=1] (orig) {} -- (\x,0) node[circle,fill,inner sep=0.7,label=above:$|\psi\rangle$] (a) {};
        \draw [dashed] (\x, 0) node[below] (A) {$\alpha$} -- (\x, \y) node[] (a) {};
        \draw [dashed] (0, \y) node[left] (B) {$\beta$} -- (\x, \y) node[] (a) {};
        
        
        % Sphere
        \draw (orig) circle (\r);
        
        % Axes
        \draw[->] (-3, 0) -- (3,0) node[right] (|1>) {$|1\rangle$};
        \draw[->] (0, -3) -- (0,3) node[above] (|0>) {$|0\rangle$};
    \end{tikzpicture}
    \caption{\label{fig:4}Promítnutí vektoru na osu změřeného výsledku.}
\end{figure}

\begin{figure}[ht]

    \centering

    \begin{tikzpicture}
        % Define radius
        \def\r{2}
        % Define x of vector
        \def\x{1.2}
        % Define y of vector
        \def\y{1.6}

        % vectors
        % shadow vector
        \draw[->, draw=pink] (0,0) node[circle,fill,inner sep=1] (orig) {} -- (\x,0) node[circle,fill,inner sep=0.7,label=above:$|\psi\rangle$] (a) {};
        % psi vector
        \draw[->,line width=0.6mm, draw=red] (0,0) node[circle,fill,inner sep=1] (orig) {} -- (\r,0) node[circle,fill,inner sep=0.7,label=above:$|\psi\rangle$] (a) {};
        \draw [dashed] (\x, 0) node[below] (A) {$\alpha$} -- (\x, \y) node[] (a) {};
        \draw [dashed] (0, \y) node[left] (B) {$\beta$} -- (\x, \y) node[] (a) {};
        
        
        % Sphere
        \draw (orig) circle (\r);
        
        % Axes
        \draw[->] (-3, 0) -- (3,0) node[right] (|1>) {$|1\rangle$};
        \draw[->] (0, -3) -- (0,3) node[above] (|0>) {$|0\rangle$};
    \end{tikzpicture}
    \caption{\label{fig:5}Prodloužení vektoru zpět na délku 1.}
\end{figure}


Schrödingerova rovnice je lineární. To znamená, že pokud za funkci $\bm{\psi}$ dosadíme součet dvou jiných vlnových funkcí ($\bm{\gamma}$ a $\bm{\epsilon}$) s libovolnými koeficienty $\bm{\alpha}$ a $\bm{\beta}$ (viz Rovnice (\ref{eq:4})), bude zachována rovnost, viz Rovnice (\ref{eq:5}). Tento součet vlnových funkcí se nazývá superpozice.


\begin{equation}
    \bm{|\psi\rangle = \alpha|\gamma\rangle + \beta|\epsilon\rangle}
    \label{eq:4}
\end{equation}

\begin{equation}
    i\hbar \frac{\partial \bm{(\alpha|\gamma\rangle + \beta|\epsilon\rangle)}}{\partial t} = -\frac{\hbar^2}{2m}
    \frac{\partial^2 \bm{(\alpha|\gamma\rangle + \beta|\epsilon\rangle)}}{\partial x^2} + V \bm{(\alpha|\gamma\rangle + \beta|\epsilon\rangle)}
    \label{eq:5}
\end{equation}

Problém měření spočívá v nelineárnosti postulátu o měření. Tuto nelineárnost může\-me jednoduše dokázat pomocí superpozice. Pokud budeme měřit částici popsanou vlnovou funkcí $|\psi\rangle$ (např. $|1\rangle$), změříme ji ve stejném stavu $|1\rangle$. Problém nastává v případě, kdy měříme částici ve stavu superpozice (např. $|\psi\rangle = \sqrt{\frac{1}{2}}|0\rangle + \sqrt{\frac{1}{2}}|1\rangle$), jelikož ji nezměříme jako tuto superpozici, ale jako jeden ze stavů superpozice ($|0\rangle$ nebo $|1\rangle$), každý s pravděpodobností 50\%.

Tato neshodnost Schrödingerovy rovnice a postulátu o měření znamená, že podle moderní kvantové mechaniky se nástroje na měření chovají podle jiných zákonů než elementární částice. Aby toto dávalo smysl museli bychom opustit předpoklad redukcionizmu, podle kterého se dá chování makroskopického objektu derivovat z chování jeho součástí. Museli bychom vytvořit teorii, která by definovala hranici, za kterou se už nemůžeme řídit redukcionizmem. Touto možností se nebudeme zaobírat, jelikož redukcionizmus je jedním z nejdůležitějších předpokladů vědy a to z dobrého důvodu\footnote[1]{Předpoklad redukcionizmu je podporován každým experimentem, který byl kdy proveden. Je těžké najít lépe podložený vědecký fakt.}. 

\subsection{Problém unifikace}
Kvantová mechanika a obecná teorie relativity jsou neslučitelné. To znamená, že tyto teorie nejsou konečné. Jejich neslučitelnost je zjevná v rozdílech fyzických reprezentací těchto teorií. 
\subsubsection{Problém času}

