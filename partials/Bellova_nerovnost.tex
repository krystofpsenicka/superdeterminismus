\section{Bellova nerovnost}
Roku 1964 zveřejnil irský fyzik John Stewart Bell vědecký článek s nadpisem \uv{ON THE EINSTEIN PODOLSKY ROSEN PARADOX}. Tímto článkem odpověděl na myšlenkový experiment předložen Albertem Einsteinem společně s Borisem Podolskym a Nathanem Rosenem. Tento myšlenkový experiment poukazuje na paradox v kvantové mechanice. Když se částice se spinem 0 rozpadne na dvě částice, podle zákonu zachování spinu musí tyto částice mít v součtu spin 0. Tomuto jevu se říká kvantové provázání. Pokud změříme spin jedné částice, instantně se dozvíme spin druhé částice, i kdyby tato částice byla vzdálena tisíce světelných let. Informace o spinu se podle kvantové mechaniky šíří mezi provázanými částicemi rychleji než světlo. Nicméně speciální teorie relativity omezuje rychlost každého kauzálního vlivu na rychlost světla. Kvantová mechanika tak porušuje lokální realismus.

Princip lokálního realismu má dvě části:
    \begin{enumerate}
        \item Princip lokality: objekt může být ovlivněn pouze jeho bezprostředním prostředím (Kauzální vliv se nemůže šířit rychleji než světlo). \parencite{lokalita}
        \item Princip realismu: vesmír existuje nezávisle na pozorovateli. \parencite{realismus}
    \end{enumerate}
    
Einstein tímto odůvodňoval svou myšlenku, že kvantová mechanika nemůže být správnou reprezentací reality. Podle něj musí každá provázaná částice nést všechny informace o svém fyzikálním stavu už od okamžiku vzniku provázání tak, že jsou tyto informace nějak skryty před vlnovou funkcí standardní kvantové mechaniky. Einstein si myslel, že musí existovat \uv{skryté proměnné}, které nejsou součástí kvantové mechaniky a spekuloval o možné teorii obsahující tyto skryté vlastnosti reality.

Bell ve svém článku zveřejnil Bellovu nerovnost, pojednávající o omezení teorie se skrytými proměnnými. Aby mohla existovat teorie splňující předpoklad lokálního realismu (teorie se skrytými proměnnými) a statistické nezávislosti, musí být splněna Bellova nerovnost (viz Rovnice \ref{eq:6}, níže), podle níž musí být korelace mezi měřením propletených částic $|S|$ menší než 2.

\begin{equation}
    |S| \leq 2
    \label{eq:6}
\end{equation}

Můžeme si představit

\subsection{Bellovy testy}
Porušení Bellovy nerovnosti bylo experimentálně dokázáno. 