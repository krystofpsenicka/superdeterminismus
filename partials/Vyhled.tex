\section{Výhled}
(Zpracováno podle článku \cite{supdet:rethink})

Do budoucna musí superdeterministický přístup vyřešit 2 problémy k tomu, aby byl úspěšný. První je experimentální. Platnost Superdeterminismu nelze ověřit pomocí Bellových testů, protože, jak už jsem vysvětlil, Superdeterminismus nesplňuje předpoklady Bellovy nerovnosti, takže jí nemusí dodržovat. Podle Superdeterminismu jsou výsledky měření v kvantové mechanice určené a ne náhodné. Dále teorie, která řeší problém měření musí být oproti kvantové mechanice nelineární\footnote[17]{Změna výstupů není úměrná změně vstupů.}, takže se v chaotickém režimu jeví pravděpodobnostně. Abychom tedy pozorovali superdeterministické odchylky od Bornova pravděpodobnostního pravidla, budeme muset provést experimenty na malých systémech v nízkých teplotách, v krátkých intervalech a ideálně na jedné a té samé částici. 

Druhý problém je na straně teoretické. Aby byl Superdeterminismus úspěšný, je nutné vyvinout obecně platnou teorii, která by vytvářela experimentálně ověřitelné předpovědi. Pokroku v této oblasti brání hlavně nedostatek úsilí. Na formulaci užitečné teorie pracuje jen málo expertů, jelikož Superdeterminismus je odsuzován jako nevědecký. Z těchto důvodů zatím existuje jen pár \uv{hracích} modelů. Doposud víme, že tato teorie musí mít podobu matematického formalismu, z něhož vyplyne nelineární evoluční zákon, který bude kontrafaktuálně neúplný a tím bude lokálně porušovat Statistickou Nezávislost. Tyto skutečnosti nás vedou k fraktální geometrii\footnote[18]{Geometrie složitě strukturovaných objektů, které se nemění při daném škálování (zvětšení nebo zmenšení).} a k p-adickým číslům, používaných v IST\footnote[19]{Invariant Set Theory: Teorie Invariantní (neměnné) Množiny}. 

\subsection{IST}
IST je jedním z nejslibnějších přístupů k Superdeterminismu. Když přemýšlíme o vzdálenosti dvou možných stavů vesmíru, intuitivně předpokládáme, že může být změřena ve stavovém prostoru\footnote[20]{Prostor, který obsahuje všechny stavy vesmíru, a jehož dimenze jsou proměnné vesmíru.} pomocí stejné Euklidovské metriky\footnote[21]{Systém měření, ve kterém se vzdálenost měří pomocí absolutní hodnoty.}, používané k měření vzdálenosti ve fyzickém prostoru. Podle tohoto přístupu je ze dvou kontrafaktuálních světů blíž k realitě ten, který se \uv{blíže} podobá realitě. Z tohoto důvodu je nelehké přistoupit na Superdeterminismus, jelikož by porušení Statistické Nezávislosti vypadalo velmi konspiračně. Toto je přesně důvod, proč si fyzikové myslí, že Kosmické Bellovy testy dokazují nemožnost Superdeterminismu. 

IST ale využívá skutečnosti, že kromě Euklidovské metriky existuje ještě p-adická metrika. P-adická metrika určuje vzdálenosti ve fraktální geometrii, stejně jako Euklidovská metrika určuje vzdálenosti v Euklidovské geometrii. Abych zbytečně nezacházel do podrobností, stačí vědět, že 2 stavy vesmíru, které jsou ve stavovém prostoru podle Euklidovské metriky velice blízké, mohou být podle p-adické metriky velice vzdálené. Např. podle 2-adické metriky je 4 dál od 5, než od 8. Fraktální geometrie má kromě p-adické metriky ještě vlastnost mnoha mezer, čímž silně omezuje kontrafaktuální světy a umožnuje jednoduché porušení Statistické nezávislosti.

IST je založena na předpokladu, že fyzikální zákony vycházejí z geometrie fraktální množiny historií, $I_U$, ve stavovém prostoru. $I_U$ představuje stavy fyzické reality, stavy realizované v našem vesmíru. Ostatní stavy vesmíru nacházející se v prostoru, ve kterém se $I_U$ nachází, jsou nemožné. Evoluční zákon přetváří vesmír z jednoho bodu v $I_U$ na jiný bod v $I_U$, z čehož vyplývá, že $I_U$ je neměnná evolučními zákony. V této teorii se jako základní prvek $I_U$ konkrétně používá fraktální helix\footnote{Šroubovice. Můžete si ho představit jako pletený provaz.} (v tom smyslu, že každá trajektorie helixu je sama sebou helixem).
